\documentclass[a4paper,11pt]{article}

% ----------------------------------------------------------------------------------------
%	FONT
% ----------------------------------------------------------------------------------------

% fontspec allows you to use TTF/OTF fonts directly
\usepackage{fontspec}
\setmainfont{Nimbus Sans}


% ----------------------------------------------------------------------------------------
%	PACKAGES
% ----------------------------------------------------------------------------------------
\usepackage{url}
\usepackage{parskip}

% other packages for formatting
\usepackage[usenames,dvipsnames]{xcolor}
\usepackage[scale=0.9]{geometry}

% tabularx environment
\usepackage{tabularx}

% for lists within experience section
\usepackage{enumitem}

% centered version of 'X' col. type
\newcolumntype{C}{>{\centering\arraybackslash}X}

% to prevent spillover of tabular into next pages
\usepackage{supertabular}
\usepackage{tabularx}
\newlength{\fullcollw}
\setlength{\fullcollw}{0.47\textwidth}

% custom \section
\usepackage{titlesec}
\usepackage{multicol}
\usepackage{multirow}

% CV Sections inspired by:
% http://stefano.italians.nl/archives/26
\titleformat{\section}{\Large\scshape\raggedright}{}{0em}{}[\titlerule]
\titlespacing{\section}{0pt}{10pt}{10pt}

% for publications
\usepackage[sorting=ydnt, minnames=2,maxnames=2]{biblatex}

% Setup hyperref package, and colours for links
\definecolor{linkcolour}{rgb}{0,0.2,0.6}

\usepackage[pdfauthor=Jimmy Aguilar Mena,
            pdftitle={CV Jimmy Aguilar Mena},
            pdfsubject={Curriculum Vitae},
            colorlinks,
            breaklinks,
            urlcolor=linkcolour,
            linkcolor=linkcolour
            ]{hyperref}
\hypersetup{}
\setlength\bibitemsep{1em}

% for social icons
\usepackage{fontawesome5}

% debug page outer frames
% \usepackage{showframe}

% Write C++ ad \CC
\def\CC{{C\nolinebreak[4]\hspace{-.05em}\raisebox{.2ex}{++}}}

% ----------------------------------------------------------------------------------------
%	BEGIN DOCUMENT
% ----------------------------------------------------------------------------------------
\begin{document}

% non-numbered pages
\pagestyle{empty}

% ----------------------------------------------------------------------------------------
%	TITLE
% ----------------------------------------------------------------------------------------

\begin{tabularx}{\linewidth}{@{} C @{}}
  \Huge{Jimmy Aguilar Mena} \\
  Barcelona, Spain\\[7.5pt]
  \href{https://github.com/Ergus}{\raisebox{-0.05\height}\faGithub~Ergus} \ $|$ \
  \href{https://https://www.linkedin.com/in/jimmy-aguilar-mena-237063167/}{\raisebox{-0.05\height}\faLinkedin~Jimmy Aguilar Mena} \ $|$ \
  \href{https://scholar.google.es/citations?user=f_W11w8AAAAJ\&hl=es}{\raisebox{-0.05\height}\faGraduationCap~Jimmy Aguilar Mena} \ $|$ \
  \href{https://join.skype.com/invite/cBUIu90wgsE8}{\raisebox{-0.05\height}\faSkype~kratsbinovish} \ $|$ \
  \href{mailto:kratsbinovish@gmail.com}{\raisebox{-0.05\height}\faEnvelope~kratsbinovish@gmail.com} \ $|$ \
  \href{tel:+34611494825}{\raisebox{-0.05\height}\faMobile~+34.611.494.825} \ $|$ \
  \faPassport~Spanish \\
\end{tabularx}


% Interests/ Keywords/ Summary
\section{Summary}
High Performance Computing Research Engineer with more than 10 years
developing hybrid parallel applications on multiple parallel
architectures.

% ----------------------------------------------------------------------------------------
\section{Work Experience}

\begin{tabularx}{\linewidth}{ @{}l r@{} }
  \textbf{General Quant Developer Associate XVA Group} & \hfill 2023-today \\
  \href{https://www.ust.com/es}{UST} External Consultant @ (\href{https://www.bbva.es/}{BBVA} Bank) &  \\

  \multicolumn{2}{@{}X@{}}{
  \begin{minipage}[t]{\linewidth}
      XVA \& Capital Group.

      \begin{itemize}[nosep,after=\strut, leftmargin=1em, itemsep=3pt]
          \item[--] Design and implement risk valuation models (CVA, FVA) for
              high performance shared and distributed heterogeneous parallel
              architectures
          \item[--] Profile, instrument and benchmark existing code to detect
              issues and optimization opportunities and modernize executions
              to take advantages of modern advanced infrastructures
          \item[--] Optimize and port prototypes to production code including
              parallelization, cache friendly and vectorization strategies
          \item[--] Mentor junior developers to implement more maintainable and
              optimized code with multiple strategies and design patterns
      \end{itemize}
  \end{minipage}
  }
\end{tabularx}


\begin{tabularx}{\linewidth}{ @{}l r@{} }
  \textbf{Software Research Engineer} & \hfill 2017--2022 \\
  Barcelona Supercomputing Center (\href{https://www.bsc.es/}{BSC}) & \\

  \multicolumn{2}{@{}X@{}}{
  \begin{minipage}[t]{\linewidth}
      Department of Computer Sciences--Microserver architectures and system software.

      \begin{itemize}[nosep,after=\strut, leftmargin=1em, itemsep=3pt]
          \item[--] Design and develop a high performance parallel task based
              programming model and its runtime system for distributed memory architectures
              (\href{https://github.com/bsc-pm/ompss-2-cluster-releases}{OmpSs2}+\href{https://github.com/bsc-pm/nanos6-cluster}{Nanos6@Cluster})
          \item[--] Hybrid Parallel Programming (distributed/shared memory,
              multi-process, multi-threading, task programming model,
              IPC, lock-free, non-blocking)
          \item[--] Parallel Application benchmark, profile, analysis, instrumentation and optimization with different tools.
              Extend the \href{https://tools.bsc.es/extrae}{Extrae} parallel profiler to provide more accurate traces for our model.
          \item[--] Language interoperability programming (C/\CC/Fortran/Python)
          \item[--] System aware application development (hwloc, Slurm API, Numa, POSIX, jemalloc)
          \item[--] Application and library development test, debug and profile automation.
          \item[--] Participation in international projects:
              \href{https://www.deep-projects.eu/}{DEEP-SEA},
              \href{https://euroexa.eu/}{EuroEXA},
              \href{https://legato-project.eu/}{LeGaTo},
              \href{https://exanode.eu/}{ExaNoDe}
      \end{itemize}
  \end{minipage}
  }
\end{tabularx}

\begin{tabularx}{\linewidth}{ @{}l r@{} }
  \textbf{CERN Openlab Summer Student} & \hfill 2015 \\
  European Organization for Nuclear Research (\href{https://home.cern/}{CERN}) & \\

  \multicolumn{2}{@{}X@{}}{
  \href{https://lhcb.web.cern.ch/}{LHCb collaboration group}.

  \begin{minipage}[t]{\linewidth}
      \begin{itemize}[nosep,after=\strut, leftmargin=1em, itemsep=3pt]
          \item[--] Kalman Filter improves using GPGPU and auto-vectorization for Online LHCb Triggers.
              \href{https://zenodo.org/record/31869}{[Link]}
          \item[--] Modify a section of the Gaudi simulator to optimize the first line
              of the filter algorithm in intense data flows conditions using
              \href{https://www.khronos.org/opencl/}{OpenCL},
              \href{https://developer.nvidia.com/cuda-zone}{Cuda} and auto-vectorization techniques.
      \end{itemize}
  \end{minipage}
  }
\end{tabularx}

\begin{tabularx}{\linewidth}{ @{}l r@{} }
  \textbf{Assistant professor of Numerical Methods and Elemental Physics} & \hfill 2011--2014 \\

  \multicolumn{2}{@{}X@{}}{
  \begin{minipage}[t]{\linewidth}
      Agrarian University of Havana
      Department of Mathematics and Physics.

      \begin{itemize}[nosep,after=\strut, leftmargin=1em, itemsep=3pt]
          \item[--] Differential equations, Numerical methods and Physics II (Thermodynamics \& Electromagnetism)
          \item[--] Python (\href{https://numpy.org/}{numpy},
              \href{https://matplotlib.org/}{matplotlib},
              \href{https://scipy.org/}{scipy}),
              \CC (\href{https://netlib.org/lapack/}{BLAS/Lapack},
              \href{https://www.gnu.org/software/gsl/}{GSL}),
              \href{https://octave.org/}{Octave},
              \href{https://maxima.sourceforge.io/}{Maxima}
          \item[--] Software development for teaching and study convergence of different numerical
              methods and dynamic systems with \href{https://doc.qt.io/qt-5.15/}{QT5},
              \href{https://plplot.sourceforge.net/}{PLplot}.
      \end{itemize}
  \end{minipage}
  }
\end{tabularx}

% ----------------------------------------------------------------------------------------
\section{Education}
\begin{tabularx}{\linewidth}{@{}l X@{}}
  2017--2022 & \textbf{PhD on Computers Architecture}
               \href{https://www.ac.upc.edu/en/academics/ph-d/ph-d-programme-by-computer-architecture}{[Link]}

               Universitat Polit\`ecnica de Catalunya (\href{https://www.upc.edu/en}{UPC}).
               Barcelona, Spain

               \textbf{Thesis:} Methodology for malleable applications on distributed memory systems.
               \href{http://paul-carpenter.org/aguilar2022thesis.pdf}{[Link]}

               \textbf{Brief description:} Extended the \href{https://pm.bsc.es/ompss-2}{OmpSs-2} task based programming model to
               work on distributed memory systems: \href{https://github.com/bsc-pm/ompss-2-cluster-releases}{OmpSs-2@Cluster}
               a runtime and model with malleability and load balancing features.
  \\

  2014--2016 & \textbf{Master Degree on High Performance Computing (MHPC)} \href{http://www.mhpc.it}{[Link]}

               Scuola Internazionale Superiore di Studi Avanzati (\href{https://www.sissa.it/}{SISSA}),
               International Center for Theoretical Physics (\href{https://www.ictp.it/}{ICTP}).
               Trieste, Italy

               Parallel programming, HPC technologies, Numerical Analysis, Monte
               Carlo Methods, Advanced Computer Architectures and Optimization,
               Parallel Linear Algebra

               \textbf{Thesis:} Enablement of a massive parallelism to study the role of disorder
               spatial correlations and anisotropies on the Anderson localization phenomenon.
               \href{https://backend.mhpc.sissa.it/sites/default/files/2021-02/JimmyAguilarMena.pdf}{[Link]}

               \textbf{Brief description:} Implement an hybrid master-slave system that exploits efficiently
               the hardware capabilities using MPI, pthreads and GPGPU programming while it offers
               simplified api and interface.
  \\
\end{tabularx}

\begin{tabularx}{\linewidth}{@{}l X@{}}
  2012--2014 & \textbf{Master Degree on Nuclear Physics}

               Higher Institute of Technologies and Applied Sciences (\href{www.instec.cu}{InSTEC}).
               La Havana, Cuba

               Advanced Statistics, Monte Carlo Methods, Big data analysis, Nuclear reactions, Nucleus Theory

               \textbf{Thesis:} Monte Carlo model implementation for secondary electron emission in Geant4 to
               simulate neutron radiography with micro-channel plates.

               \textbf{Brief description:} Implement a new physical model to the
               \href{https://geant4.web.cern.ch/}{Geant4} \CC framework optimized with multi-threading
               and auto-vectorization. Use \href{https://root.cern/}{ROOT} and
               \href{https://www.r-project.org/}{R} to automatize processing the big amount of
               generated data.
  \\
  2006--2011 & \textbf{Bachelor Degree on Nuclear Physics}

               Higher Institute of Technologies and Applied Sciences (\href{www.instec.cu}{InSTEC}).
               La Havana, Cuba

               \textbf{Thesis:} Monte Carlo validation for a $\Delta E-E$ spectrometer for angular
               distribution measurements of Nuclear Reactions with heavy ions with
               \href{https://geant4.web.cern.ch/}{Geant4}.
  \\
\end{tabularx}

% ----------------------------------------------------------------------------------------
\section{Skills}
\begin{tabularx}{\linewidth}{@{}l X@{}}
  Language & Spanish (Native), English (Advanced), Italian (Basic), Catalan (Basic)\\
  Operative System & GNU/Linux (User and kernel space development) BSD (User space development), \\
  Programming & \textbf{Advanced}: \href{https://www.gnu.org/software/bash/}{Bash},
                C\,(99--17), \CC\,(98--20, STL, \href{https://www.boost.org/}{Boost}), Fortran, Lisp,
                \href{https://www.freepascal.org/}{Pascal}, PHP,
                Python\,(\href{https://numpy.org/}{numpy},
                \href{https://matplotlib.org/}{matplotlib},
                \href{https://pandas.pydata.org/}{pandas},
                \href{https://docs.python.org/3/c-api/index.html}{python c-api},
                \href{https://scipy.org/}{SciPy})

                \textbf{Basic-Intermediate}: Java,
                \href{https://www.perl.org/}{Perl},
                \href{https://www.rust-lang.org/}{Rust},
                SQL,
                (\href{https://mariadb.org/}{MariaDB},
                \href{https://www.postgresql.org/}{PostgreSQL} and
                \href{https://www.sqlite.org/index.html}{SQLite}), Visual Basic  \\

  Other Technologies & \textbf{Advanced}:
                       Autotools,
                       \href{https://cmake.org/}{CMake}/CTest/CPack,
                       \href{https://developer.nvidia.com/cuda-zone}{Cuda},
                       \href{https://www.docker.com/}{Docker},
                       \href{https://doxygen.nl/}{Doxygen},
                       \href{https://tools.bsc.es/extrae}{Extrae}/\href{https://tools.bsc.es/paraver}{Paraver},
                       \href{https://www.sourceware.org/gdb/}{gdb},
                       \href{https://git-scm.com/}{git},
                       \href{https://developers.google.com/protocol-buffers}{Google Protobuf},
                       \href{https://github.com/google/googletest}{Google Test},
                       IPC\,(Posix, Sysv,
                       \href{https://lwn.net/Articles/405284/}{CMA},
                       \href{https://knem.gitlabpages.inria.fr/}{KNEM},
                       \href{https://github.com/hpc/xpmem}{XPMEM}),
                       \href{https://www.mpi-forum.org/}{MPI},
                       \href{https://www.kernel.org/doc/html/v4.19/vm/numa.html}{Numa},
                       \href{https://pm.bsc.es/ompss-2}{OmpSs-2},
                       \href{https://www.khronos.org/api/opencl}{OpenCL},
                       \href{https://www.openmp.org/}{OpenMP},
                       \href{https://perf.wiki.kernel.org/index.php/Main_Page}{Perf}/\href{https://icl.utk.edu/papi/}{PAPI},
                       \href{https://www.qt.io/}{Qt (QtWidget/QML)},
                       \href{https://slurm.schedmd.com/}{slurm},
                       \href{https://valgrind.org/}{valgrind}

                       \textbf{Basic-Intermediate}:
                       \href{https://github.com/features/actions}{Github Actions},
                       \href{https://google.github.io/googletest/}{GoogleTest},
                       \href{https://grpc.io/}{grpc},
                       \href{https://www.jenkins.io/}{Jenkins},
                       \href{https://www.atlassian.com/es/software/jira}{Jira},
                       \href{https://kubernetes.io/}{Kubernetes},
                       \href{https://www.openacc.org/}{OpenACC},
                       \href{https://www.qemu.org/}{QEMU}/\href{https://www.linux-kvm.org/page/Main_Page}{KVM},
                       \href{https://www.travis-ci.com/}{Travis} \\
\end{tabularx}

% ----------------------------------------------------------------------------------------
\section{Publications}
\begin{refsection}[citations.bib]
    \nocite{*}
    \printbibliography[heading=none]
\end{refsection}


\vfill
\center{\footnotesize Last updated: \today}

\end{document}
%%% Local Variables:
%%% mode: latex
%%% TeX-master: t
%%% TeX-engine: luatex
%%% End:
