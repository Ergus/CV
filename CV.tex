\documentclass[a4paper,10pt]{article}

% ----------------------------------------------------------------------------------------
%	FONT
% ----------------------------------------------------------------------------------------

% fontspec allows you to use TTF/OTF fonts directly
% \usepackage{fontspec}

% ----------------------------------------------------------------------------------------
%	PACKAGES
% ----------------------------------------------------------------------------------------
\usepackage{url}
\usepackage{parskip}

% other packages for formatting
\usepackage[usenames,dvipsnames]{xcolor}
\usepackage[scale=0.9]{geometry}

% tabularx environment
\usepackage{tabularx}

% for lists within experience section
\usepackage{enumitem}

% centered version of 'X' col. type
\newcolumntype{C}{>{\centering\arraybackslash}X}

% to prevent spillover of tabular into next pages
\usepackage{supertabular}
\usepackage{tabularx}
\newlength{\fullcollw}
\setlength{\fullcollw}{0.47\textwidth}

% custom \section
\usepackage{titlesec}
\usepackage{multicol}
\usepackage{multirow}

% CV Sections inspired by:
% http://stefano.italians.nl/archives/26
\titleformat{\section}{\Large\scshape\raggedright}{}{0em}{}[\titlerule]
\titlespacing{\section}{0pt}{10pt}{10pt}

% for publications
\usepackage[sorting=ydnt, minnames=2,maxnames=2]{biblatex}

% Setup hyperref package, and colours for links
\definecolor{linkcolour}{rgb}{0,0.2,0.6}

\usepackage[pdfauthor=Jimmy Aguilar Mena,
            pdftitle={CV Jimmy Aguilar Mena},
            pdfsubject={Curriculum Vitae},
            colorlinks,
            breaklinks,
            urlcolor=linkcolour,
            linkcolor=linkcolour
            ]{hyperref}
\hypersetup{}
\setlength\bibitemsep{1em}

% for social icons
\usepackage{fontawesome5}

% debug page outer frames
% \usepackage{showframe}

% ----------------------------------------------------------------------------------------
%	BEGIN DOCUMENT
% ----------------------------------------------------------------------------------------
\begin{document}

% non-numbered pages
\pagestyle{empty}

% ----------------------------------------------------------------------------------------
%	TITLE
% ----------------------------------------------------------------------------------------

\begin{tabularx}{\linewidth}{@{} C @{}}
    \Huge{Jimmy Aguilar Mena} \\[7.5pt]
    \href{https://github.com/Ergus}{\raisebox{-0.05\height}\faGithub\ Ergus} \ $|$ \
    \href{https://scholar.google.es/citations?user=f_W11w8AAAAJ\&hl=es}{\raisebox{-0.05\height}\faGraduationCap\ Jimmy Aguilar Mena} \ $|$ \
    \href{https://https://www.linkedin.com/in/jimmy-aguilar-mena-237063167/}{\raisebox{-0.05\height}\faLinkedin\ Jimmy Aguilar Mena} \ $|$ \
    \href{mailto:kratsbinovish@gmail.com}{\raisebox{-0.05\height}\faEnvelope \ kratsbinovish@gmail.com} \ $|$ \
    \href{tel:+34611494825}{\raisebox{-0.05\height}\faMobile \ +34.611.494.825} \\
\end{tabularx}



% Interests/ Keywords/ Summary
\section{Summary}
High Performance Computing Research Engineer with more than 10 years
developing parallel applications. Interested in new technologies,
challenging architectures, frameworks and systems.

% ----------------------------------------------------------------------------------------
\section{Work Experience}

\begin{tabularx}{\linewidth}{ @{}l r@{} }
  \textbf{Research Engineer \& PhD Student} & \hfill 2017 - present \\

  \multicolumn{2}{@{}X@{}}{
  \begin{minipage}[t]{\linewidth}
      Department of Computer Sciences--Microserver architectures and system software.
      Barcelona Supercomputing Center (\href{https://www.bsc.es/}{BSC})

      \begin{itemize}[nosep,after=\strut, leftmargin=1em, itemsep=3pt]
          \item[--] Parallel Runtime System Development (\href{https://github.com/bsc-pm/nanos6-cluster}{Nanos6@Cluster})
          \item[--] Hybrid Parallel Programming (shared and distributed memory, multiprocess, multithreading, task programming model)
          \item[--] Parallel Application Profile, Analysis, Instrumentation and Optimization
          \item[--] Language interoperability programming (C/C++/Fortran/Python)
          \item[--] System aware application development (hwloc, Slurm API, Numa, POSIX, jemalloc)
          \item[--] Participation in international projects:
              \href{https://www.deep-projects.eu/}{DEEP-SEA},
              \href{https://euroexa.eu/}{EuroEXA},
              \href{https://legato-project.eu/}{LeGaTo},
              \href{https://exanode.eu/}{ExaNoDe}
      \end{itemize}
  \end{minipage}
  }
\end{tabularx}

\begin{tabularx}{\linewidth}{ @{}l r@{} }
    \textbf{Assistant professor of Numerical Methods and Elemental Physics} & \hfill 2011 - 2014 \\[3.75pt]
    \multicolumn{2}{@{}X@{}}{
      \begin{minipage}[t]{\linewidth}
          \begin{itemize}[nosep,after=\strut, leftmargin=1em, itemsep=3pt]
              \item[--] Differential equations, Numerical methods and Physics II (Thermodynamics \& Electromagnetism)
              \item[--] Python (\href{https://numpy.org/}{numpy},
                  	\href{https://matplotlib.org/}{matplotlib},
                  	\href{https://scipy.org/}{scipy}),
                  C++ (\href{https://netlib.org/lapack/}{BLAS/Lapack},
                  	\href{https://www.gnu.org/software/gsl/}{GSL}),
                  \href{https://octave.org/}{Octave},
                  \href{https://maxima.sourceforge.io/}{Maxima}
          \end{itemize}
      \end{minipage}
    }
\end{tabularx}

% ----------------------------------------------------------------------------------------
\section{Education}
\begin{tabularx}{\linewidth}{@{}l X@{}}
  2017--2022 & \textbf{PhD on Computers Architecture}
               \href{https://www.ac.upc.edu/en/academics/ph-d/ph-d-programme-by-computer-architecture}{[Link]}

               Universitat Polit\`ecnica de Catalunya (\href{https://www.upc.edu/en}{UPC}).
               Barcelona, Spain

               \textbf{Thesis:} Methodology for malleable applications on distributed memory systems.
               \href{http://paul-carpenter.org/aguilar2022thesis.pdf}{[Link]}

               \textbf{Brief description:} Extended the \href{https://pm.bsc.es/ompss-2}{OmpSs-2} task based programming model to
               work on distributed memory systems: \href{https://github.com/bsc-pm/ompss-2-cluster-releases}{OmpSs-2@Cluster}
               a runtime and model with malleability and load balancing features.
  \\

  2015 & \textbf{CERN Openlab Summer Student} \href{https://home.cern/science/computing/cern-openlab}{[Link]}

         European Organization for Nuclear Research (\href{https://home.cern/}{CERN}).
         Geneve, Switzerland.

         \textbf{Work:} Kalman Filter improves using GPGPU and auto-vectorization for Online LHCb Triggers.
         \href{https://zenodo.org/record/31869}{[Link]}

         \textbf{Brief description:} Modify a section of the Gaudi simulator to optimize the first line
         of the filter algorithm in intense data flows conditions using \href{https://www.khronos.org/opencl/}{OpenCL},
         \href{https://developer.nvidia.com/cuda-zone}{Cuda} and auto-vectorization techniques.
  \\

  2014--2017 & \textbf{Master Degree on High Performance Computing (MHPC)} \href{http://www.mhpc.it}{[Link]}

               Scuola Internazionale Superiore di Studi Avanzati (\href{https://www.sissa.it/}{SISSA}),
               International Center for Theoretical Physics (\href{https://www.ictp.it/}{ICTP}).
               Trieste, Italy

               Parallel programming, HPC technologies, Numerical Analysis, Monte
               Carlo Methods, Advanced Computer Architectures and Optimization,
               Parallel Linear Algebra

               \textbf{Thesis:} Enablement of a massive parallelism to study the role of disorder
               spatial correlations and anisotropies on the Anderson localization phenomenon.
               \href{https://backend.mhpc.sissa.it/sites/default/files/2021-02/JimmyAguilarMena.pdf}{[Link]}

               \textbf{Brief description:} Implement an hybrid master-slave system that exploits efficiently
               the hardware capabilities using MPI, pthreads and GPGPU programming while it offers
               simplified api and interface.
  \\

  2012--2014 & \textbf{Master Degree on Nuclear Physics}

               Higher Institute of Technologies and Applied Sciences (\href{www.instec.cu}{InSTEC}).
               La Havana, Cuba

               Advanced Statistics, Monte Carlo Methods, Big data analysis, Nuclear reactions, Nucleus Theory

               \textbf{Thesis:} Implementing a model for secondary electron emission in Geant4 to
               simulate neutron radiography with micro-channel plates.

               \textbf{Brief description:} Implement a new physical model to the
               \href{https://geant4.web.cern.ch/}{Geant4} C++ framework optimized with multi-threading
               and auto-vectorization. Use \href{https://root.cern/}{ROOT} and
               \href{https://www.r-project.org/}{R} to automatize processing the big amount of
               generated data.
  \\

  2006--2011 & \textbf{Bachelor Degree on Nuclear Physics}

               Higher Institute of Technologies and Applied Sciences (\href{www.instec.cu}{InSTEC}).
               La Havana, Cuba

               \textbf{Thesis:} Monte Carlo validation for a $\Delta E-E$ spectrometer for angular
               distribution measurements of Nuclear Reactions with heavy ions with
               \href{https://geant4.web.cern.ch/}{Geant4}.
  \\
\end{tabularx}

% ----------------------------------------------------------------------------------------
\section{Skills}
\begin{tabularx}{\linewidth}{@{}l X@{}}
  Language & Spanish (Native), English (Advanced), Italian (Basic), Catalan (Basic)\\
  Programming & Intermediate-Advanced: C, C++, Fortran, Python, Bash,
                Pascal, Cuda, OpenCL, PHP, Lisp

                Basic-Intermediate: Rust, C\#, Java, Visual Basic,
                SQL (PostgreSQL, MariaDB, SQLite) \\

  Other Technologies & Advanced: OpenMP, MPI, OmpSs-2, Numa, Qt, Perf (PAPI),
                       gdb, valgrind, git, Autotools, CMake/CTest/CPack

                       Basic-Intermediate: Github Actions, Travis, grpc,
                       GoogleTest, Grafana, Arduino. \\
\end{tabularx}

% ----------------------------------------------------------------------------------------
\section{Publications}
\begin{refsection}[citations.bib]
    \nocite{*}
    \printbibliography[heading=none]
\end{refsection}


\vfill
\center{\footnotesize Last updated: \today}

\end{document}
%%% Local Variables:
%%% mode: latex
%%% TeX-master: t
%%% End:
